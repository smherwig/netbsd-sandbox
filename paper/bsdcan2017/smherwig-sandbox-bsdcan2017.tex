\documentclass[letterpaper,twocolumn,9pt]{article}
\usepackage[utf8]{inputenc}
\usepackage{csquotes}
\usepackage[english]{babel}
\usepackage{usenix,epsfig,endnotes}

\usepackage[backend=biber]{biblatex}
\addbibresource{smherwig-sandbox-bsdcan2017.bib}

\begin{document}

%don't want date printed
\date{}

%make title bold and 14 pt font (Latex default is non-bold, 16 pt)
\title{\bf secmodel\_sandbox : An application sandbox for NetBSD}

%for single author (just remove % characters)
\author{
{\rm Stephen Herwig} \\
University of Maryland, College Park
} % end author

\maketitle

% Use the following at camera-ready time to suppress page numbers.
% Comment it out when you first submit the paper for review.
\thispagestyle{empty}

% - remove abstract from title
% - add section headers
% - add a section that provides an overview of kauth(9) and secmodel(9).
%
% The LWC paper has the format:
%   Abstract (2 paragraphs)
%   Introduction        do
%   Related Work        do, possibly merge into introduction
%   NetBSD Background   Try to keep this section small
%   Design              this is like your sandbox API
%   Implementation      can have a section on kaut background here 
%   Evaluation          instead, talk about applications
%   Conclusions         
%   Acknowledgements    save for last version


\subsection*{Abtract}
We introduce a new security model for NetBSD -- secmodel\_sandbox -- that
allows per-process policies for restricting privileges.   Privileges correspond to
kauth authorization requests, such as a request to create a socket or read a
file, and policies specify the sandbox's decision: deny, defer, or allow.
Processes may apply multiple sandbox policies to themselves, in which case the
policies stack, and child processes inherit their parent's sandbox.  Sandbox
policies are expressed in Lua, and the evaluation of policies uses NetBSD 7's
experimental in-kernel Lua interpreter.  As such, policies may express static
authorization decisions, or may register Lua functions that secmodel\_sandbox
invokes for a decision.

\section{Introduction}
A process sandbox is a mechanism for limiting the privileges of a process, as
in restricting the operations the process may perform, the resources it may
use, or its view of of the system.  Sandboxes address the dual problems of
limiting the potential damage caused by running an untrusted binary, and
mitigating the effects of exploitation of a trusted binary.  In either case,
the goal is to restrict a process to only the necessary privileges for the
purported task, and, in the latter case, to also drop privileges when they are
no longer needed.

% a comment on the strengths and weaknesses of theses tools (both technical and
% non-technical) -- e.g., TOCTOU
Although NetBSD currently lacks a sandbox mechanism, sandbox implementations
exist for various operating systems.  systrace \cite{SystracePaper}, a
multi-platform mechanism used in earlier versions of NetBSD, and
seccomp \cite{SeccompOverview}, a Linux-specific implementation,
exemplify the approach of specifying a per-process system call policy, and use
system call interposition to enforce the policy filter.  For systrace, the
policy format is systrace-specific, whereas seccomp specifies the policy as a
BPF program.  OpenBSD's pledge system call \cite{PledgeManpage} offers a
simplified interface for dropping privileges: OpenBSD groups the POSIX
interface into categories, and allows processes to whitelist or \emph{pledge}
their use of certain categories; an attempt to perform an operation from a
non-pledged category kills the process.  

We implement an application sandbox for NetBSD, secmodel\_sandbox, that allows
per-process restriction of privileges.  secmodel\_sandbox plugs into the kauth
framework, and uses NetBSD's support for in-kernel Lua
\cite{ScriptableOperatingSystems} to both specify and evaluate sandbox
policies.  We are developing several facilities with secmodel\_sandbox, such as
a secure chroot and a partial emulation of OpenBSD's pledge system call.

\section{NetBSD Overview} 

\subsection{kauth}
NetBSD 4.0 introduced the kauth kernel subsystem
\cite{NetBSDSecurityEnhancements} -- a clean room implementation of Apple's
kauth framework \cite{MacKauth} for OS X -- to handle authorization requests
for privileged operations.  Privileged operations are represented as triples of
the form \emph{(scope, action, optional subaction)}.  The predefined scopes are
\texttt{system}, \texttt{process}, \texttt{network}, \texttt{machdep},
\texttt{device}, and \texttt{vnode}, each forming a namespace that is further
refined by the action and subaction components.  For instance, the operation
to create a socket is identified by the triple \texttt{(network, socket,
open)}, and the operation to read a file by \texttt{(vnode, read\_data)}.   

Some authorizations, such as \texttt{(process, nice)}, are triggered by a single
system call (setpriority); some, such as \texttt{(system, mount, update)}, are
triggered when a system call (mount) is called with specific arguments (the
\texttt{MNT\_UPDATE} flag); and others, such as \texttt{(system, filehandle)}
may be triggered by more than one system call (fhopen and fhstat).  Many system
calls do not trigger a kauth request.

kauth uses an observer pattern whereby listeners register for operation
requests for a given scope; when a request occurs, each listener is called.

Each listener receives as arguments the operation triple, the
credentials of the object (typically, the process) that triggered the
authorization request, as well as additional context specific to the
request.

Each listener returns a decision: either \texttt{allow},  \texttt{deny}, or
\texttt{defer}.  If any listener returns \texttt{deny}, the request is denied.
If at least one listener returns \texttt{allow} and none returns \texttt{deny},
the request is allowed.  If all listeners return \texttt{defer}, the decision
is scope-dependent.  For all scopes other than the vnode scope, the result is
to deny the authorization.  For the vnode scope, the authorization request
contains a ``fall-back" decision, which nearly always specifies a decision
conforming to traditional BSD4.4 file access permissions.

\subsection{secmodel}
While the NetBSD kernel source contains many listeners (typically in accordance
with kernel configuration options), the secmodel framework offers a lightweight
convention for developing and managing a set of listeners that represents a
larger security model.  By default, NetBSD uses secmodel\_bsd44, which
implements the traditional security model based on 4.4BSD, and which itself is
composed of three separate models: secmodel\_suser, secmodel\_securelevel, and
secmodel\_extensions.

An important, subtle point with the default security model is that many
authorization requests are deferred, relying on kauth's default behavior when
all listeners return \texttt{defer} to fully implement the policy.  

\section{Design}
We developed secmodel\_sandbox as a loadable kernel module with companion
user-space library \texttt{libsandbox}.  By convention, we install the device
file for secmodel\_sandbox at \texttt{/dev/sandbox}.

A process interacts with secmodel\_sandbox via the \texttt{sandbox(const char
*script, int flags)} function of \texttt{libsandbox}.  The argument
\texttt{script} is a Lua script that specifies the sandbox policy.  The
\texttt{flag} argument specifies the action to take when a process attempts a
denied operation: a value of \texttt{0} means that the operation returns an
appropriate errno as dictated by kauth (typically \texttt{EACCES} for
kauth's vnode scope and \texttt{EPERM} for all other scopes); a value of
\texttt{SANDBOX\_ON\_DENY\_KILL} specifies the pledge behavior of killing
the process.  The \texttt{sandbox} function packages these arguments into a
struct and, via an ioctl call, passes the struct to \texttt{/dev/sandbox}.

secmodel\_sandbox evaluates the Lua script in a Lua environment that is
pre-populated with a \texttt{sandbox} Lua module.  The \texttt{sandbox} Lua
module allows a script to set policy rules via the following interface:

\begin{verbatim}
  sandbox.default(result)
  sandbox.allow(req)
  sandbox.deny(req)
  sandbox.on(req, func)
\end{verbatim}

The \texttt{sandbox.default} function specifies a result of either
\texttt{`allow'}, \texttt{`deny'}, or \texttt{`defer'}.  The result is the
sandbox's decision for any kauth request for which the script does not specify
a more specific rule.

The \texttt{sandbox.allow} and \texttt{sandbox.deny} specify allow and deny
rules, respectively, for the kauth request given as \texttt{req}.  

The sandbox Lua module uses strings of the form
\texttt{`scope.action.subaction'} to represent the requests; hence, a request
to open a socket corresponds to the string \texttt{`network.socket.open'}, and
a request to read a file to \texttt{`vnode.read\_data'}.  A script may specify a
complete request name, or a prefix.  When the process triggers an authorization
request, secmodel\_sandbox will select the policy rule that has the longest
prefix match with the given request.  As an example, a sandbox policy script
of:
\begin{verbatim} 
  sandbox.default(`deny') 
  sandbox.allow(`network')
\end{verbatim} 
would allow any request in the network scope, but would deny requests from all
other scopes.

The \texttt{sandbox.on} Lua function registers a Lua function \texttt{func} to
be called for the given kauth request.  The signature for \texttt{func} is:
\begin{verbatim}
  func(req, cred, arg0, arg1, arg2, arg3)
\end{verbatim}
where \texttt{req} is the kauth request that generated the callback, cred is a
Lua table that represents the credentials of the requesting object or process,
and the remaining arguments are request-specific.  All parameters for
\texttt{func} exist only in the Lua environment; manipulating the values does
not affect the underlying C objects that they represent.

For many requests, the values for \texttt{arg0} through \texttt{arg3} are
\texttt{nil}, as the kauth request carries no additional context.  For the
requests that do contain context, we translate the context into appropriate Lua
values.  For example, for the request \texttt{`network.socket.open'}, the
arguments are Lua integers representing the arguments to the \texttt{socket}
system call that triggered the request.  For clarity in script writing, we
pre-populate the \texttt{sandbox} Lua module with symbols for common constants,
such as \texttt{sandbox.AF\_INET} and \texttt{sandbox.SOCK\_STREAM}.\  For
requests in the process scope, \texttt{arg0} is a Lua table that represents a
subset of the fields of the \texttt{struct proc} that is the target of the
request, such as the \texttt{pid}, \texttt{ppid}, \texttt{comm} (program name),
and \texttt{nice} value.  Callback functions for the vnode scope receive as
\texttt{arg0} a Lua table that contains the pathname and file status
information of the target vnode.  Completely representing the context with Lua
values is an ongoing effort.

\section{Sandbox Implementation}
% TODO: provide an introduction to the rest of the section here that explains
% the main tricks or surprises that informed the implementation.
%
%   1. efficiency - having a ruleset rather than searching in Lua every time
%   2. allowing rules to be dynamic - in effect, for the sandbox.on function
%       to create new rules during its evaluation.
%   3. minding the subtelies of the default secuirty model so as not to
%      accidentally elevate privileges.
%   5. additional safegurads to ensure that sandboxes are isolated (and hence,
%      cannot be undone).
%   4. changing/extending the normal behavior of fork in order to ensure that
%      sandboxes are inherited, but that parent and child continue to add new
%      sandboxes independent of the other.

Our design and implementation of secmodel\_sandbox considered several important
requirements and features.  First, while expressing rules in Lua is elegant,
having to call into Lua to find a matching rule for each request is not.  Thus,
we implemented secmodel\_sandbox so thatevaluating the policy script
``compiles'' the rules into a prefix tree, mimicking the natural hierarchy
provided by the (scope, action, subaction) format of requests.  Thus,
secmodel\_sandbox can quickly find a matching rule, and only needs to call
into Lua for \emph{functional} rules -- rules specified as Lua functions via
\texttt{sandbox.on}.

Second, we wanted to allow sandboxes to be \emph{dynamic}; that is, allow a
functional rule to set other rules.  For example, a script might create rules
based on the requesting credential, as in the following, which installs a
functional rule for the network scope so that different rules may be created
for the root user and for ordinary users:
\begin{verbatim}
  sandbox.on(`network', function(rule, cred)
    if cred.euid == 0 then
        sandbox.allow(`network.bind')
        ...
    else
        sandbox.deny(`network.bind')
        ...
    end
  end)
\end{verbatim}

Third, we had to be mindful of the subtleties of the default security model,
particularly its dependency on kauth's default decisions when all listeners
return defer, so as not to allow sandboxes to elevate a process's privileges
beyond the default model.  In a similar vein, we needed to isolate multiple
sandboxes on a single process so that the process is not able to install a new
sandbox that loosens or undoes a rule of an existing sandbox.

Finally, in order to ensure that child processes inherit the sandboxes of their
parent, but that, after process creation, parent and child may apply additional
sandboxes independently of one another, we had to extend the normal forking
behavior.

\subsection{Sandbox creation}
When a process sets a sandbox policy via \texttt{libsandbox}, the kernel
creates a new sandbox, represented as a \texttt{struct sandbox}. A sandbox
contains two main items: a Lua state and a ruleset.  The Lua state is the Lua
environment in which secmodel\_sandbox evaluates all Lua code for that
particluar sandbox.  The ruleset is a prefix tree that secmodel\_sandbox
searches during a kauth request to find the sandbox's matching rule. 

Before secmodel\_sandbox evaluates the policy script in the newly created Lua
state, secmodel\_sandbox adds the sandbox Lua functions (e.g.,
\texttt{sandbox.allow}) and constants (e.g., \texttt{sandbox.AF\_INET}) to the
state.  Each sandbox Lua function is a closure that contains a pointer to the
\texttt{struct sandbox}.  In Lua terminology, the \texttt{struct sandbox }is a
light userdata upvalue.  

When the script calls a sandbox Lua function, the function -- which is
implemented in C code -- performs argument checking, retrieves the ruleset from
the \texttt{struct sandbox} upvalue, and inserts the rule and the rule's value
into the ruleset.  

For \texttt{sandbox.allow}, \texttt{sandbox.deny}, and
\texttt{sandbox.default}, the rule's value is a trilean: one of
\texttt{KAUTH\_RESULT\_ALLOW}, \texttt{KAUTH\_RESULT\_DENY}, or
\texttt{KAUTH\_RESULT\_DEFER}, as defined in \texttt{sys/kauth.h}.  For
\texttt{sandbox.on}, the value is an index into Lua's registry.  The Lua
registry is a global table that can only be accessed from C code.  When a
script invokes \texttt{sandbox.on}, secmodel\_sandbox stores the callback
function at an unused index in the Lua registry, and the ruleset stores this
index as the rule's value.

After evaluating the policy script, secmodel\_sandbox attaches the
\texttt{struct sandox} to the current process's credentials.  The data that
secmodel\_sandbox attaches to a credential is in fact a list of \texttt{struct
sandbox}'s, to support allowing a process to apply multiple sandboxes during
the course of its execution.  If the list does not exist, secmodel\_sandbox
first creates it and inserts the new sandbox; otherwise, the new \texttt{struct
sandbox} is added to the existing list.

Storing \texttt{struct sandbox} as an upvalue supports the creation of
dynamic rules; that is, a \texttt{sandbox.on} callback function that
creates rules for other requests as part of its evaluation.  If the callback
function creates new rules by calling any of the \texttt{sandbox} Lua module
functions, then the C implementations of these functions can immediately find
the corresponding ruleset for the given Lua state.

\subsection{Evaluating Authorization Requests}
% XXX we might need to call kauth_cred_setdata in the lua C-functions.
secmodel\_sandbox registers listeners for all kauth scopes.  When one of the
secmodel\_sandbox listeners is called, the listener checks whether a list of
\texttt{struct sandbox}s is attached to the requesting credential.  If a list
is not attached, the listener defers; if a list is attached, secmodel\_sandbox
searches the ruleset of each \texttt{struct sandbox} for a value, calling into
Lua if the value represents a registry index for a callback function.   

If any sandbox in the list returns \texttt{deny}, secmodel\_sandox returns deny
for the request; if at least one sandbox returns \texttt{allow} and none
returns \texttt{deny}, secmodel\_sandbox returns \texttt{defer}, not
\texttt{allow} as one would presume.  The reason for ``converting"
\texttt{allow} to \texttt{defer} is due to subtleties in the implementation of
kauth(9) and of the default security models that implement the traditional
BSD4.4 security policy.  In particular, since a large part of the traditional
security model is implemented by having all listeners defer, and thus relying
on kauth's ``fall-back" behavior, secmodel\_sandbox must also defer, so as not
to allow the elevation of privileges.

\subsection{Process forking}
% sys/kern/kern_fork.c::sys_fork
%   sys/kern/kern_fork.c::fork1
%       sys/kern/kern_auth.c::kauth_proc_fork
In NetBSD, a process's credentials are represented by the
\texttt{kauth\_cred\_t} type.  The kauth framework emits events
corresponding to a credential's life-cycle via the \texttt{cred} scope.  As with
other kauth scopes, listeners may register callback functions.

When a process forks, the normal behavior is for the parent and child to share
the same \texttt{kauth\_cred\_t}, and to simply increment the credential's
reference count.  During the fork process, however, the kauth framework
emits a \texttt{fork} event, thereby allowing for other behavior.  For the
\texttt{fork} event, the listener callback functions receive the \texttt{struct
proc} of the parent and child, as well as the shared credential.

secmodel\_sandbox registers a callback for credential events.  During a
\texttt{fork} event, secmodel\_sandbox checks whether the credential contains a
list of sandboxes.  If yes, then secmodel\_sandbox creates a new credential for
the child process that is identical to the parent's credential, with the
exception that the child credential contains a new list head for the list of
sandboxes.  Althought the list head of the parent and child are different, they
both point to the same initial \texttt{struct sandbox}.  Thus, each sandboxed
process has its own \texttt{kauth\_cred\_t} and its own sandbox list head, but
the individual \texttt{struct sandbox}s are shared among the related processes,
and hence reference counted.

% add footnote
The handling of sandboxes in this manner means that the child is restricted by
the same sandboxes as its parent at the time of the child's creation, but that
after the child's creation, parent and child may add additional sandbox
policies that do not affect the other process.

\subsection{Mapping vnodes to pathnames}
The request context for the the \texttt{vnode} kauth scope contains the vnode
that is the target of the operation.  For a sandbox policy, however, it is much
more natural to work with pathnames rather than vnodes.

secmodel\_sandbox uses methods similar to those in
\texttt{sys/kern/vfs\_getcwd.c} to attempt to retrieve a pathname.  The method
is to search for the basename of the vnode in the \texttt{namei} cache via
\texttt{cache\_revlookup}, and then walk back to the root vnode via
interspersing calls to \texttt{VOP\_LOOKUP} (to retrieve a parent vnode),
and \texttt{VOP\_READDIR} (to find the path name component of the child
vnode).   While we would expect the initial vnode to be present in the cache,
an obvious weakness of this method is the reliance on a cache hit, which cannot
be gauranteed.

\subsection{Safeguards}
Evaluating user-provided Lua scripts in the kernel raises a few concerns.  An
obvious concern is denial-of-service caused by a Lua script with an infinite
loop.  While not yet implemented, the defense is straight-forward, and
used in the Lua kernel module to handle creating Lua states with \texttt{luactl}.

In short, as part of its C API, Lua provides the function
\texttt{lua\_sethook} for an application to register a C hook function
to be called at various Lua VM events.  In particular, an application
can register to receive a callback after every $n$ Lua VM instructions.  The
approach is therefore to set a hook function to be called after some maximum
number of VM instructions; if the hook is called, the hook stops execution of the
Lua VM by calling \texttt{lua\_error}.  Lua allows only one hook function per
Lua state; in order to ``restore'' the VM instruction count back to zero, the
hook function must be reset before every evaluation of a Lua script or
function.

Another concern is that the \texttt{struct sandbox}s or the callbacks
registered via \texttt{sandbox.on} might be accessed and modified from Lua
code.  Values in the Lua registry and upvalues are, provided Lua's
\texttt{debug} library is not loaded, only accessible from C code.
secmodel\_sandbox does not load the \texttt{debug} library.  Moreover,
secmodel\_sandbox does not provide a \texttt{require} function or any other
means to load additional Lua libraries.

\section{Applications}
In this section, we describe the tools and facilities we are developing with
secmodel\_sandbox.

\subsection{Secure chroot}
One application that we are developing is a secure chroot.  In 2011, Aleksey
Cheusov proposed the secmodel\_securechroot security model
\cite{SecmodelSecureChroot}.  secmodel\_securechroot was developed as a kernel
module, and once loaded, modifies the chroot system call to place additional
restrictions on the chrooted process.  The restrictions impose process
containment by preventing process's with one root directory from viewing
information about processes with a different root directory, as well as denying
the chrooted process several system-wide operations, such as rebooting,
modifying sysctls, or adding devices. 

On NetBSD's \texttt{tech-kern} mailing list, there was disagreement over the
exact operations that should be allowed and denied within a secure chroot.
Moreover, some users expressed a desire to not override the default
\texttt{chroot} behavior, but rather have an additional system call for secure
chroot, so that users could choose the level of restriction for each chrooted
process.  While some of the changes to kauth needed to support
\texttt{secmodel\_sandbox} were merged into the NetBSD kernel source, the
secmodel itself was not.

We are developing an implementation of secmodel\_securechroot as an auxiliary
function, \texttt{sandbox\_securechroot}, in \texttt{libsandbox}, with an
associated command-line tool.  Development of the tool demonstrates that
previously proposed secmodels can be implemented using secmodel\_sandbox, and
that secmodel\_sandbox offers users flexibility in choosing the proper level of
containment.

%In 2008, the Gaols project \cite{GaolsPaper} attempted to implement FreeBSD's jail
%facility in NetBSD using the kauth(9) framework.  The project stressed
%the limits of kauth(9), and required a number of modifications to kauth(9).
%The more interesting changes involved allowing the binding of a socket to only
%an approved list of addresses for the specific jail. in paritcular replacing
%certain newtork addresses with an address from the list of local network
%addresses permitted in the prision.  When an imporsioned process reqeust to
%bind a socket to the unspecified orloopback address, or tries to connect othe
%loopback address, the jail code needs to fix up the address with an address
%form the list of local network addresses permitted in the jail.

\subsection{pledge}
% one small exception is wpath's fstat()
We are also developing the \texttt{libsandbox} auxiliary function
\texttt{sandbox\_pledge}, which  attempts to emulate OpenBSD's pledge system
call using secmodel\_sandbox.

A sandbox policy that mimics pledge is essentially a whitelist: explicitly
allowing actions that correspond to a given category, and denying all others.
Certain categories are easily implemented.  For instance, the pledge
categories that correspond to the access and modification of file metadata,
such as \texttt{rpath}, \texttt{wpath}, \texttt{fattr}, and \texttt{chown},
are, with small exceptions, handled by appropriate \texttt{vnode} scope rules.
Similarly, categories that limit network access to certain domains, such as
\texttt{inet} and \texttt{unix}, are covered by rules for
\texttt{`network.bind'} and \texttt{`network.socket.open'}.

Several pledge categories, however, reference system calls that, in NetBSD, do
not trigger kauth requests.  For example, the \texttt{flock} category that
allows file locking or the \texttt{dns} category that allows DNS network
transactions, lack apprporiate kauth requests.  As a result such categories
cannot be implemented.

\section{Conclusion}
% Reiterate what you have done?
% What have you shown?
% What was hard?
% What do you think will be easy?
% What is the advantage of this?
We have introduced and developed a new security model, secmodel\_sandbox, for
NetBSD that allows per-process specification and restriction of privileges.
While several secmodels exist for NetBSD, secmodel\_sandbox is novel in its use
of NetBSD's in-kernel Lua interpreter to allow processes to express privileges,
subject to the bounds of the traditional BSD4.4 security model.  We designed
secmodel\_sandbox to limit excessive calls into Lua, to allow sandboxes to
dynamically create rules during the execution of a process, to allow a process
to specify multiple, isolated, sandboxes during the course of its execution,
and to ensure that a child process inherits the sandbox of its parent.  We are
developing concrete, familiar, applications in order to demonstrate our
design's ease and flexibility in developing secure software.


% FUTURE WORK
% ===========
% using fileassoc() and veriexec() to attach sandbox policies to executables
% implementing secmodel_rbac
%
% perhaps mention that in a later section you will comment on what can be done 
% if changes to the system are allowd.

% sandbox_device.c::sandbox_device_setspec
%   sandbox.c::sandbox_attach
%       kern_auth.c::kauth_cred_getdata
%       sandbox.c::sandbox_create
%           sandbox_ruleset.c::sandbox_ruleset_create
%           sandbox_lua.c::sanbox_lua_newstate
%           sandbox_lua.c::sandbox_lua_load
%       either secmodel_sandbox_attachcurproc, or kauth_cred_setdata

\printbibliography

\end{document}
